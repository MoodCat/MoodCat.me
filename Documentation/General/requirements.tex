\documentclass{scrreprt}
\usepackage{listings}
\usepackage{underscore}
\usepackage[bookmarks=true]{hyperref}
\hypersetup{
    pdftitle={Software Requirement Specification},    % title
    colorlinks=false,       % false: boxed links; true: colored links
    linktoc=page            % only page is linked
}%
\def\myversion{1.0 }
\title{%
\flushright
\rule{16cm}{5pt}\vskip1cm
\Huge{SOFTWARE REQUIREMENTS\\ SPECIFICATION}\\
\vspace{2cm}
for\\
\vspace{2cm}
MoodCat\\
\vspace{2cm}
\LARGE{Release 1.0\\}
\vfill
\rule{16cm}{5pt}
}
\date{}
\usepackage{hyperref}
\usepackage[top=2 cm, bottom= 2 cm, left=1 cm, right=1 cm]{geometry}


\begin{document}
\maketitle



\chapter*{Functional requirements}
\begin{table}[h]
\begin{tabular}{|p{1cm}|p{3cm}|p{14cm}|}
\textbf{\#} & \textbf{Name} & \textbf{Description} \\
      1.0   & \textbf{\textit{Website}} &  \\
   	  1.1   & front page & When the user comes at the website the user gets a screen to sign in.
   	   If the user is done or the logging in is completed the user is redirected to the choose mood page.  \\
      1.2   & choose mood page & On this page the user can choose it's mood or click the "what's my mood" button.\\
      1.3   & what's my mood & When the user isn't sure what mood he is in, he can do a quiz in which he finds out in which room he should be placed. \\
      1.4  	& chat room & When a user has chosen it's mood it comes in a chat room to listen music together. 
      		 There then can chat about the music. \\   
      		 
      2.0 	& \textbf{\textit{Chat room}} & \\
	  2.1   & play & In the chat room there plays music depending on the mood of the room. This is al music from soundcloud.\\
	  2.2   & vote to skip & When a user doesn't like this music it can ask to skip the music. If enough people want to skip this, the number is skipped. \\
	  2.3   & switch & Switch from chat room if the room doesn't fits your mood. \\
	  2.4   & split  & Split the chat room in 2 if there are 2 groups of people depending on skips. \\
	  3.0	& \textbf{\textit{Learning Algorithm}} & \\
	  3.1   & perceptron & We use a multilayer perceptron to predict the mood of a song. \\
	  3.2   & recommender system & We use a recommender system to predict if a song should be played in a certain room. \\  
      4.0   & \textbf{\textit{Gamification}} &  \\
      4.1   & achievements  & An user gets achievements if he fulfils the requirement.     \\
      4.2   & awards  & At a certain amount of contribution you get awards in terms of adding a song to a room and more. This certain amount is to be determined.            
\end{tabular}
\end{table}
	
\chapter*{Non-functional requirements}





\end{document}