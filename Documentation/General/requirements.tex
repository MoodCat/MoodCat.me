\documentclass{scrreprt}
\usepackage{listings}
\usepackage{underscore}
\usepackage[bookmarks=true]{hyperref}
\hypersetup{
    pdftitle={Software Requirement Specification},    % title
    colorlinks=false,       % false: boxed links; true: colored links
    linktoc=page            % only page is linked
}%
\def\myversion{1.0 }
\title{%
\flushright
\rule{16cm}{5pt}\vskip1cm
\Huge{SOFTWARE REQUIREMENTS\\ SPECIFICATION}\\
\vspace{2cm}
for\\
\vspace{2cm}
MoodCat\\
\vspace{2cm}
\LARGE{Release 1.0\\}
\vfill
\rule{16cm}{5pt}
}
\author{MoodCat.me}
\date{}
\usepackage{hyperref}
\usepackage[top=2 cm, bottom= 2 cm, left=1 cm, right=1 cm]{geometry}


\begin{document}
\maketitle



\chapter*{Functional requirements}
\begin{table}[h]
\begin{tabular}{|p{0.5cm}|p{3.5cm}|p{14cm}|}
\textbf{\#} & \textbf{Name} & \textbf{Description} \\
         & \textbf{\textit{Website}} &  \\
   	  1.1   & front page & When the user is not logged in and visits the homepage, he should be redirected to the log-in screen.
   	   When the user is logged in, the user is redirected to the mood-selection page.  \\
      1.2   & mood-selection page & On this page the user can select his mood. When the user selected his mood, he gets redirected to a chat room with this mood.\\
      1.3  	& chat room & When the user enters a chat room, he gets songs served that are for his current mood. \\   
      		 
       	& \textbf{\textit{Chat room}} & \\
	  2.1   & listening & The chat room plays music depending on the mood of the room. The songs are streamed from SoundCloud.\\
	  2.2   & vote to skip & When a user doesn't like the song currently streamed in the chat room, he can vote to skip the song. If enough people want to skip the current song, the next song is streamed to all users in the room. \\
	  2.3   & user management  & If there are too many people in a chat room, half of the users are redirected to a new chat room. \\
	  	& \textbf{\textit{Learning Algorithm}} & \\
	  3.1   & perceptron & We use a multilayer perceptron to predict the mood of a song. The perceptron is trained on song-data received from EEMCS.\\
	  3.2   & recommender system & We use a recommender system to select the next song that should be played in a chat room. \\  
         & \textbf{\textit{Gamification}} &  \\
      4.1   & achievements  & The user can earn achievements if he fulfils various requirements related to the achievement.     \\
      4.2	& user contribution & The user can contribute to the recommender system and receive points in return. \\ 
      4.3   & awards  & At a various levels for user points, the user receives awards.
\end{tabular}
\end{table}
	
\chapter*{Non-functional requirements}

\begin{table}[h]
\begin{tabular}{|p{0.5cm}|p{3.5cm}|p{14cm}|}
\textbf{\#} & \textbf{Name} & \textbf{Description} \\
	  1.1 & versioning & Git will be used as a versioning system.\\
	  1.2 & pulling & For every new feature a pull request is made. When a pull request has fulfilled all the requirements below it's accepted.      \\
      1.3 & test coverage & At all times a test coverage of more than 80\% must be maintained on the core. On the frond-end we try to test as much as possible. We will do exhaustive manual testing on every new feature.\\
      1.4 & Checkstyle warnings & Whenever a Pull request is submitted, it is not allowed to add more than 5 Checkstyle warnings compared to master.\\
      1.5 & continuous integration & Maven and Travis CI will be used for continuous integration and can't fail.\\
      1.6 & Javadoc on classes & Every class must be javadoc-ed witch summarizes the responsibilities and refers to all classes it depends on.\\
      1.7 & Javadoc on methods & Every method must be javadoc-ed witch tells the responsibilities,parameters and return value of the function.\\
      1.8 & issues & Github issues must be used in combination with waffle.io (witch uses github issues).\\
\end{tabular}
\end{table}





\end{document}