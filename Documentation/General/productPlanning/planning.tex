\chapter{Product}

\section{High-level product backlog}
The MoSCoW model is used to structure the high-level product backlog.
The items are listed according to the priority in relation to the other items in the backlog.
The backlog is translated to an actual list of issues on \gls{GitHub} (which is ordered using Waffle\cite{waffle}).
\documentclass{scrreprt}
\usepackage{listings}
\usepackage{underscore}
\usepackage[bookmarks=true]{hyperref}
\hypersetup{
    pdftitle={Software Requirement Specification},    % title
    colorlinks=false,       % false: boxed links; true: colored links
    linktoc=page            % only page is linked
}%
\def\myversion{1.0 }
\title{%
\flushright
\rule{16cm}{5pt}\vskip1cm
\Huge{SOFTWARE REQUIREMENTS\\ SPECIFICATION}\\
\vspace{2cm}
for\\
\vspace{2cm}
MoodCat\\
\vspace{2cm}
\LARGE{Release 1.0\\}
\vfill
\rule{16cm}{5pt}
}
\author{MoodCat.me}
\date{}
\usepackage{hyperref}
\usepackage[top=2 cm, bottom= 2 cm, left=1 cm, right=1 cm]{geometry}


\begin{document}
\maketitle



\chapter*{Functional requirements}
\begin{table}[h]
\begin{tabular}{|p{0.5cm}|p{3.8cm}|p{13.4cm}|}
\textbf{\#} & \textbf{Name} & \textbf{Description} \\
         & \textbf{\textit{Website}} &  \\
   	  1.1   & Front page & When the user is not logged in and visits the homepage, he will be redirected to the log-in screen, else the user is redirected to the mood-selection page.  \\
      1.2   & Mood-selection page & On this page the user can select his mood. When the user selected his mood, he gets offered a list of compatible rooms.\\
      1.3	& Joining a room & The User can now pick a room and join it. \\
      1.4  	& Room & When the user enters a room, he is served songs based on his chosen mood. \\   
      		 
       	& \textbf{\textit{Room}} & \\
	  2.1   & Listening & The songs are streamed from SoundCloud.\\
	  2.2   & Vote to skip & When a user doesn't like the song currently streamed in the room, the user can vote to skip the song. If enough users want to skip the current song, the next song is streamed to all users in the room. \\
	  2.3   & User management  & Users periodically receive suggestions of the system to join a different room. \\
	  	& \textbf{\textit{Learning Algorithm}} & \\
	  3.1   & Perceptron & We use a multilayer perceptron to predict the mood of a song. The perceptron is trained on song metadata received from EEMCS.\\
	  3.2   & Recommender system & A recommender system will select the next song that should be played in a room. \\  
         & \textbf{\textit{Gamification}} &  \\
      4.1   & Rewards  & At a various levels for user points, the user receives rewards.     \\
      4.2	& User contribution & The user can contribute to the recommender system and receive points in return. \\
\end{tabular}
\end{table}
	
\chapter*{Non-functional requirements}

\begin{table}[h]
\begin{tabular}{|p{0.5cm}|p{3.5cm}|p{13.4cm}|}
\textbf{\#} & \textbf{Name} & \textbf{Description} \\
	  1.1 & Versioning & Git will be used as a versioning system.\\
	  1.2 & Pull Requests & For every completed feature a pull request is created. A pull request will be merged to the master branch if it fulfills all non-functional requirements below. \\
      1.3 & Test coverage & At all times, the test coverage on the core must be more than 80\%. Every new feature needs manual testing.\\
      1.4 & Checkstyle warnings & Whenever a pull request is submitted, it is not allowed to add more than 5 Checkstyle warnings compared to the master branch.\\
      1.5 & Continuous \newline Integration & Travis CI will be used for continuous integration. The master branch builds are never allowed to fail.\\
      1.6 & Documentation & Every package, class and method must have Javadoc. Angular controllers must have NgDoc.\\
      1.7 & Issues & Github issues must be used in combination with waffle.io.\\
\end{tabular}
\end{table}





\end{document}

\section{Roadmap}
The roadmap below shows a general overview of the goals for each \gls{SCRUM} sprint in a week.
This is an early planning and is expected to change.

\begin{itemize}
\item Week 1: Product setup\\
This week we will setup our development environments and start brainstorming.

\begin{itemize}
\item Set up GitHub project \cite{githubRepo}
\item Create initial idea of MoodCat
\end{itemize}

\item Week 2: Sprint 1\\
This week we will focus on setting up a Graphical User Interface to let the user interact with the product.
We also concretized our idea further, and wrote an initial product vision.

\begin{itemize}
\item Set up initial version of frontend.
\item Write product vision draft.
\end{itemize}

\item Week 3: Sprint 2\\
This week we will set up a connection between the frontend and backend system.

\begin{itemize}
\item Create various APIs on the backend.
\item Develop frontend to backend connection.
\item Develop interface on frontend to use \gls{APIs}.
\end{itemize}

\item Week 4: Sprint 3\\
The first version of the music-matching algorithm will be created. \footnote{We found out making the music matching algorithm was not feasible. We discussed with our stakeholder and decided to let it go.}

\begin{itemize}
\item Choose songs to use as test data.
\item Create testcases to measure the performance of our algorithm.
\item Design the neural network. 
\item You can vote how good or bad a song is.
\end{itemize}

\item Week 5: Sprint 4\\
The connection between the frontend and backend will be adjusted. We will get a basic overview of to to create an algorithm that will provide song suggestions for the room.

\begin{itemize}
\item Make API's for the frontend (chat and general room).
\item Connection between API's and backend.
\item Make a user classification system.
\item Use nearest neighbour to find rooms that suits a mood.
\end{itemize}

\item Week 6: Sprint 5\\
The music-matching algorithm will be made. We also review our own code and test quality, since we're half way.

\begin{itemize}
\item Review code and test quality.
\item Make the matching algorithm between users and rooms.
\end{itemize}

\item Week 7: Sprint 6\\
We will apply \gls{SIG} feedback and implement an auto room creation system.

\begin{itemize}
\item Apply \gls{SIG} feedback.
\item Implement auto room creation system.
\end{itemize}

\item Week 8: Sprint 7\\
Finish documentation and start final report.

\begin{itemize}
\item Finish architecture design document.
\item Create final report.
\end{itemize}


\item Week 9: Sprint 8\\
Finish final report and prepare presentation.

\begin{itemize}
\item Finish final report.
\item Prepare and practise presentation.
\end{itemize}


\end{itemize}

\chapter{Product backlog}

\section{User stories of features}
As a user,\\
When I start the app,\\
Then I should see a mood-selection page.

As a user,\\
When I select a mood,\\
Then I should see a list of rooms.

As a user,\\
When I select a room,\\
Then the current song should be played and I should see a chat box with the latest messages that have been sent in this room .

As a user,\\
When I type in the chat message input field,\\
And I click on 'Send message',\\
Then the message should be send to the backend.

As a user,\\
When a user in my current room sends a message to the backend,\\
Then I should see that message in the chat box.

As a user,\\
When I click on the 'thumbs up' or 'thumbs down' buttons,\\
Then the system should alter according to the feedback to enhance the song-selection algorithm.

As a user,\\
When I click on the 'mute' button,\\
Then the music should be muted.


\section{Initial release plan}
Each sprint has a corresponding milestone. For example, if a feature will be developed during sprint 3, the corresponding GitHub issue will have milestone 3. This makes sure that the roadmap outlines are closely followed and progress can be easily determined by using the GitHub tools.

\chapter{Definition of Done}
We need to set a definition of done.
This entails that for every feature, sprint and the end product, we will define what we consider "done".\\

A feature is completed when it is implemented according to its specification and the code is well tested using unit tests.
Furthermore, the public API (the public methods) of introduced classes should be well-documented using \gls{JavaDoc} or \gls{NgDoc} and follow our \gls{checkstyle} and markup rules.
Finally, the feature must be reviewed by at least one team member that was not directly involved in creating the code and tests.
If all these prerequisites are met, the pull request can be merged.\\

We will use sprint iterations of one week. All stories assigned for a sprint should be done at the end of a sprint.
If a story could not be completed in the sprint, it will be copied to the next sprint and this will be discussed in the sprint reflection in order to prevent planning failures in the future.
The system also has to be manually tested for the main interaction with the system. Lastly, integration tests are created and ran successfully.\\

The final product is done when all requirements are fulfilled and the implementation matches the specification. The implementation should be well tested using unit tests and preferably integration / end-to-end tests as well.
If a should-have requirement is not implemented, this has to be agreed upon by all team members with a valid reason, and the product should still be usable without the feature.
The code base is formatted and documentated in a consistent manner and validated by the \gls{SIG} test