
\subsection{Functional requirements}
For the product MoodCat, the requirements regarding functionality and service are grouped under the \emph{Functional requirements}.
Within these functional requirements, four categories can be identified using the \emph{MoSCoW} model \footnote{\url{http://en.wikipedia.org/wiki/MoSCoW_method}} for prioritizing requirements:

\subsubsection{Must haves}
\begin{itemize}
\item The user shall interact with the product via a web interface.

\item The web interface shall show a screen to log in to the application.

\item The login mechanism shall be managed by SoundCloud OAuth keys ("Log in with SoundCloud").

\item The login mechanism shall show an error page when the user does not authorize MoodCat to have access to their SoundCloud data.

\item The web interface shall show a first-use tutorial upon logging in successfully for the first time.

\item The web interface shall show the user a page where they can select their current mood.

\item The web interface shall show the following moods:
 Nervous, Exiting, Happy, Pleasing, Relaxing, Peaceful, Calm, Sleepy, Bored and Sad. \footnote{This list has been chosen, because it's well spread over the grid. This is also a popular list in literature for example \cite{Book}}

\item The system shall serve the user a list of rooms, ordered by how close they match the mood of the user.

\item The web interface shall present the user with a room when it is chosen, which consists of a chat and music player.
	
\item The platform should be able to stream music to the client.

\item The system shall use a multi-layer perceptron system to predict the mood of a song.

\item The system will use a recommender system to determine the next song to be played in each room.

\end{itemize}
\subsubsection{Should haves}
\begin{itemize}
\item Users in the same room listen to the same song at the same point in time.

\item A room shall contain a chat, connecting people with the same mood together.

\item A room shall contain a now playing slider.

\item The web interface shall give the user the option to mute or disconnect the current stream.

\item The web interface shall notify the user if they made an illogical combination of moods, warning them results may vary.

\item A room has a button to up-vote the current song, indicating the song is a good fit to the room.

\item A room has a button to down-vote the current song, indicating the song is a bad fit to the room. This will also vote to skip the song.

\item A song shall be skipped in one of the following situations:
	\begin{itemize}
	\item The song received $\frac{\$number\_of\_users\_in\_room}{3}$ votes to skip.
	\item The song received at least 10 votes, and the ratio upvote:downvote is 1:2 or less.
	\end{itemize}

\item The multi-layer perceptron will be initialized by generating values from the metadata in our dataset.

\item The upvotes and downvotes for a song change the weights of the multi-layer perceptron.

\item The user shall be rewarded for up- and downvoting with "treats". These treats are achievement points.

\item The system shall retrieve more metadata about a song from AcousticBrainz\footnote{\url{http://acousticbrainz.org/}}.

\end{itemize}

\subsubsection{Could haves}
\begin{itemize}
\item The system shall provide a search engine that aids the user in finding a song to add.

\item The web interface shall notify the user if they made an illogical combination of moods, warning them results may vary.

\item If SoundCloud provides MoodCat with a purchase\_url \footnote{\url{https://developers.soundcloud.com/docs/api/reference\#tracks}}, This will be made available to the user.

\item With enough achievement points the user can receive predefined perks, such as the ability to add a number to the room.

\item The system shall periodically suggest the user to hop to another room to encourage more interaction and we can direct the user to a more preferred mood.

\item If metadata is not present in the AcousticBrainz database the system shall run an analysis itself using the AcousticBrainz software.

\item Visualisation for the current mood


\item Show artist trivia information for the now playing song.

\end{itemize}

\subsubsection{Would/Won't haves}
\begin{itemize}
\item We will focus on the development of a web interface, and will not build native clients for any type of device.

\item The service won't have social network integration.

\item The chat won't have a chatbot.

\item The service won't have a list of listeners.

\item The service won't have trending topics or moods.

\end{itemize}

\subsection{Non-functional requirements}
\begin{itemize}

\item For the project, the Scrum methodology shall be applied.

\item Code shall be versioned using Git

\item Git shall be used with the PR methodology: Every new feature will be developed in a branch, once completed it will be turned into a pull request. This will be reviewed and improved, and if it is accepted the branch gets merged into master.

\item Front-end and Back-end shall communicate in a RESTful way, using an API.

\item The product shall consist of a separate front end and backend repo, ensuring proper RESTful implementation is used.

\item The backend will be implemented in Java. The front-end web app will be built in \gls{HTML}, \gls{CSS} and \gls{JavaScript}, using the \gls{AnuglarJS} framework. 

\item The project shall use Continuous Integration to ensure the entire codebase is tested for every change. In this project, \gls{Travis CI} will be the tool of choice.

\item Code readability shall be safeguarded with the use of Checkstyle and FindBugs. Code formatting will also be agreed upon to prevent auto-formatting issues in git diffs.

\item Core systems shall be unit tested with a coverage of at least 80\%. In addition, integration tests can be implemented.

\item Issues shall be tracked using waffle.io\footnote{\url{http://www.waffle.io}}. This ensures a Scrum style tracking of issues, while still having the overview on github.

\item Every package, class and method shall be documented using JavaDoc. Angular controllers must be documented using NgDoc. 

\end{itemize}
