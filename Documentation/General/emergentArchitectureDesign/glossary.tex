\usepackage[toc,style=treenoname,order=word,subentrycounter]{glossaries}
\makeglossaries

\newglossaryentry{HTML}{
  name= HTML,
  description={HyperText Markup Language is the standard markup language used to create web pages}
}

\newglossaryentry{CSS}{
  name= CSS,
  description={Cascading Style Sheets is a style sheet language used for describing the look and formatting of a document written in a markup language}
}

\newglossaryentry{JS}{
  name= Javascript,
  description={a dynamic programming language. It is most commonly used as part of web browsers, whose implementations allow client-side scripts to interact with the user, control the browser, communicate asynchronously, and alter the document content that is displayed}
}

\newglossaryentry{HTTP}{
  name= HTTP,
  description={The Hypertext Transfer Protocol is an application protocol for distributed, collaborative, hypermedia information systems}
}

\newglossaryentry{JSON}{
  name= JSON,
  description={JavaScript Object Notation, is an open standard format that uses human-readable text to transmit data objects consisting of attribute–value pairs. It is used primarily to transmit data between a server and web application, as an alternative to XML}
}

\newglossaryentry{SQL}{
  name= SQL,
  description={Structured Query Language  is a special-purpose programming language designed for managing data held in a relational database management system (RDBMS), or for stream processing in a relational data stream management system (RDSMS)}
}

\newglossaryentry{ARIA}{
  name= Aria,
  description={Accessible Rich Internet Applications defines ways to make Web content and Web applications (especially those developed with Ajax and JavaScript) more accessible to people with disabilities}
}

\newglossaryentry{RDB}{
  name= Relational database,
  description={A relational database is a digital database whose organization is based on the relational model of data, as proposed by E.F. Codd in 1970. This model organizes data into one or more tables (or "relations") of rows and columns, with a unique key for each row}
}

\newglossaryentry{ORM}{
  name= Object-relational mapping,
  description={A programming technique for converting data between incompatible type systems in object-oriented programming languages}
}

\newglossaryentry{Package}{
  name= package,
  plural=packages,
  description={A Java package is a mechanism for organizing Java classes into namespaces similar to the modules of Modula, providing modular programming in Java}
}

\newglossaryentry{Class}{
  name= class,
  plural=classes,
  description={An extensible program-code-template for creating objects, providing initial values for state (member variables) and implementations of behavior (member functions, methods)}
}

\newglossaryentry{Interface}{
  name= interface,
  plural= interfaces,
  description={a common means for unrelated objects to communicate with each other. These are definitions of methods and values which the objects agree upon in order to cooperate}
}

\newglossaryentry{ContinuousIntegration}{
  name= Continuous Integration,
  description={ the practice of merging all developer working copies with a shared mainline several times a day}
}

\newglossaryentry{library}{
  name= library,
  plural= libraries,
  description={a collection of non-volatile resources used by computer programs}
}

\newglossaryentry{framework}{
  name= framework,
  plural= frameworks,
  description={an abstraction in which software providing generic functionality can be selectively changed by additional user-written code, thus providing application-specific software}
}

\newglossaryentry{OOP}{
  name= Object-oriented programming,
  description={programming paradigm based on the concept of "objects", which are data structures that contain data, in the form of fields, often known as attributes; and code, in the form of procedures, often known as methods}
}
