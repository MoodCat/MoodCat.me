\documentclass[10pt,a4paper]{article}
\usepackage[utf8]{inputenc}
\usepackage{amsmath}
\usepackage{amsfonts}
\usepackage{amssymb}
\begin{document}

\title{Product Vision\\
	\begin{small}
	The why, who and what
	\end{small}
}
\author{MoodCat.me}
\maketitle

\section{Why should we build MoodCat?}
Listening to music is currently a solo activity.
Various competitors provide a system that can offer users popular songs or songs that are similar to songs the user already listened to.
These systems don't offer users songs they haven't heard yet and that might fit the user's current need.
Since these systems provide solely songs that fit only the user and are very limited in varience, a better solution can be achieved.

Secondly there are some competitors that offer users a room to socialize while experiencing songs selected by other users in the room.
While this might offer the user unknown songs, the system is user-driven and does not offer the listeners that want to listen on their own.

\section{What are the selling points of MoodCat?}
MoodCat has several selling points that makes it unique in the current music market.

\begin{enumerate}
\item MoodCat is self-maintaining, which makes it possible to handle audience sizes from one to one thousand.
This makes it possible to use the system no matter how many other users are currently online.

\item MoodCat offers users both to experience solely or communicate with fellow music listeners.

\item MoodCat does not suffer a lot of a cold start.
The system is able to offer relevant songs using the specified current mood of the user.

\item MoodCat offers unknown/lesser known artists the possibility to promote their songs.
The system will choose songs that are in the long tail of fewer-played songs in order to keep a high variety of songs.

\item MoodCat can determine mood trends of a long-term user and can adjust the song-selection mechanism accordingly.
Therefore long-term users are served more suitable songs which increases their overall satisfaction.
\end{enumerate}

\section{What is the target audience of MoodCat?}
Our target audience consists of (but not limiting to) the following persons:

\begin{enumerate}
\item Everyone who wants to listen to music without the hassle of searching suitable songs for their current mood/preferences.

\item Unknown/lesser known artists that created songs they want to offer the system user to listen to.

\item Everyone who wants to discover and listen to songs they haven't heard yet.

\item Everyone who wants to listen to and experience songs and share this experience with fellow listeners in the same online room.

\item Everyone who wants to listen at any time on the day, no matter how many users are currently online.
\end{enumerate}

\section{Which components should MoodCat have?}
MoodCat should have the following components in order to have a shippable product:

\begin{enumerate}
\item MoodCat should have a self-maintainable song-selection system that can offer users suitable songs for their current mood.

\item MoodCat should have a page that offers users the possibility to communicate with fellow users.

\item Moodcat should offer users the ability to express their current mood in order to influence the song-selection mechanism.
\end{enumerate}

\end{document}