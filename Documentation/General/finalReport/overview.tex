\chapter{Overview of the developed and implemented software product}

\section{Introduction}
MoodCat is a website where people can listen to music together, it combines enjoying your mood with social interaction.
When entering MoodCat, a user can choose one or more moods from a given list. Based on these moods a list of rooms
sorted on relevance is shown, displaying their names and the song currently playing. When entering a room
the user can chat with other users that have joined the room.

The software of MoodCat consists of two main components: the Java based backend and the frontend which is a website,
hosted at \url{http://moodcat.me}.

\section{Frontend}
The frontend uses Angular.js for managing its objects and connection with the backend and bootstrap
for the layout and design.
Angular.js is an MVC framework written in Javascript, next to providing a nice way to split
and integrate with models, views and controllers, it also simplifies managing and updating displayed data.
Angular reads HTML pages that contain custom tag attributes. The attributes are interpreted by angular, and
replaced with actual data from the model.
The model is in turn updated through constant communication with the backend, which is made possible by the REST API of the backend,
and angular's capability of parsing it. The current song and the time of the music is synced with the backends so that
all users in the rooms always get to hear the same sound, simultaneously.

\section{Backend}
The backend is responsible for managing the underlying structure of the project, to
connect loosely coupled with the frontend and to persist and retrieve data from our database.
The backend connects through hibernate to a postgres database running on a server which is also hosted on \url{http://moodcat.me}, next to an Nginx server for the
frontend and the backend running as Java application. The database contains metadata of songs, artists, rooms and users. When the
backend is started, it connects with the database and retrieves the current rooms. The REST API is then initialized, which then
handles calls at \url{http://moodcat.me/api/}, a typical API call would be a GET request to \url{http://moodcat.me/api/rooms/} to obtain the currently
active rooms.
The backend contains models, of which the fields are sent through the API as generated JSON, instances, which contain not only metadata but also
dynamically changing fields like the current time of songs, and entities, which are direct representations of the entries in the database.

\section{Mapping moods}
For comparing rooms, moods and songs we use valence arousal vectors. Every mood can be mapped by a 2 dimensional vector of valence and arousal,
both of which have values between -1 and 1. Each room has its own vector which is used to determine which room to suggest based on
the selection of moods. When rooms need new songs, the backend will seek new songs that have vectors close to the room's vector, this is done
through R-Trees.
The songs are mapped to moods through a user feedback system. When a user likes the song that plays in the room,
the song can be upvoted and our system will know this song fits in the room.
When a user downvotes a song, the song apparently doesn't fit into the room, the user will now be asked for a more detailed
classification: the user can choose a degree of both valence and arousal, and the valence arousal vector is slightly updated accordingly.
