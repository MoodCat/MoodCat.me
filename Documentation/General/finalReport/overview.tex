\chapter{Overview of the developed and implemented software product}

MoodCat is a website where people can listen to music together.
It combines enjoying your mood with social interaction.
When entering MoodCat, the user can choose one or more moods from a given list.
Based on the selected moods, a list of rooms sorted on relevance is shown, displaying the names and the song currently playing in the rooms. 
When entering a room the user can chat with other users that have joined the room.

\section{Components}
The software of MoodCat consists of two main components: the Java based backend and the frontend which is a website,
hosted at \url{http://moodcat.me}.

\subsection{Frontend}
The frontend uses \gls{angularjs} for managing its objects and connection with the backend and bootstrap for the layout and design.
\gls{angularjs} is an \gls{MVVM} framework written in Javascript.
It provides a mechanism to separate concerns between models, views and controllers.
It also simplifies managing and updating displayed data through two-way \gls{data-binding}.
\gls{angularjs} parses \gls{HTML} elements that contain custom tag attributes.
The attributes are interpreted by \gls{angularjs} and replaced with actual data from the model.
The model is updated through constant communication with the backend.
The current song and the time of the music is synced with the backends in order to make sure the users in a room listen to the same song at the same time.

\subsection{Backend}
The backend is responsible for managing the underlying structure of the project, to connect loosely coupled with the frontend and to persist and retrieve data from our database.
The backend connects through \gls{Hibernate} to a \gls{Postgres} database running on a server which is also hosted on the same server.
The database contains the songs, artists, rooms and users.
When the backend is started, it connects with the database and retrieves the current rooms.
The \gls{REST} API is then initialized, which can handle calls at \url{http://moodcat.me/api/}.
The backend provides models, of which the fields are sent through the \gls{REST} API as generated \gls{JSON}.
The models represent the inner content of the entities in the database.

\section{Mapping moods}
\label{mapping-moods}
In order to compare rooms, moods and songs we use \gls{valence} \gls{arousal} \textbf{vectors}.
Every mood is mapped by a 2-dimensional vector of \gls{valence} \gls{arousal}, both of which have values between $-1$ and $1$.
Each room has its own \textbf{vector} which is used to determine which room to suggest based on the selection of moods.
In order to queue up new songs for a room, the backend will fetch new songs that have vectors close to the room's vector.
The database uses a \gls{rtree} spatial index to quickly calculate distances and order the songs accordingly.
The indexing method is further described in the appendix \ref{appendix:rtree}.
The songs are mapped to vectors through a user feedback system.
When a user likes a song that plays in the room, the song can be upvoted.
The system then knows this song fits in the room.
When a user downvotes a song, the song apparently does not fit the room.
The user will now be asked to provide a classification for the song.
The user can choose a degree of both valence and arousal and the valence arousal vector is slightly tweaked accordingly.
