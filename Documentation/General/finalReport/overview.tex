\chapter{Overview of the developed and implemented software product}

\section{Introduction}
When entering MoodCat, a user can choose multiple of a list of moods to listen music to. Based on these moods a list of rooms
sorted on relevance is shown, along with its title and which song is currently playing. When entering a room
the user can chat with other users that have joined the room.

The software of MoodCat consists of two main components: the Java based backend and the frontend which is a website,
hosted at \url{http://moodcat.me}.

\section{Frontend}
The frontend uses Angular.js for managing its objects and connection with the backend and bootstrap
for the layout and design.
Angular.js is an MVC framework written in Javascript, next to providing a nice way to split
and integrate with models, views and controllers, it also simplifies managing and updating displayed data.
Angular reads HTML pages that contain custom tag attributes. The attributes are interpreted by angular, and
replaced with actual data from the model.
The model is in turn updated through constant interval based polling of the backend, which is made possible by the REST API from the backend
and angular's capability of parsing it. When the time of the application is too much out of sync with the backend,
the frontend will know and can correct it. When a song has finished the frontend will retrieve the metadata of the new song in the room, the model
will change and the displayed page is updated accordingly, no need to load a new page.

\section{Backend}
The backend is responsible for managing the underlying structure of the project, to
connect loosely coupled with the frontend and to persist and retrieve data from our database.
The backend connects through hibernate to a postgres database running on a server which is also hosted on \url{http://moodcat.me}, next to an Nginx server for the
frontend and the backend running as Java application. The database contains metadata of songs, artists, rooms and users. When the
backend is started, it connects with the database and retrieves the current rooms. The REST API is then initialized, which then
handles calls at \url{http://moodcat.me/api/}, a typical API call would be a GET request to \url{http://moodcat.me/api/rooms/} to obtain the currently
active rooms.
The backend contains models, of which the fields are sent through the API as generated JSON, instances, which contain not only metadata but also
dynamically changing fields like the current time of songs, and entities, which are direct representations of the entries in the database.
Instances can be created through entities, and models use instances to populate their fields.

\section{Mapping moods}
For comparing rooms, moods and songs we use valence arousal vectors. Every mood can be mapped by a 2 dimensional vector of valence and arousal,
both of which have values between -1 and 1. Each room has its own vector which is used to determine which room to suggest based on
the selection of moods. When rooms need new songs, the backend will seek new songs that have vectors close to the room's vector, this is done
through R-Trees.
The songs are mapped to moods through a user feedback system.
