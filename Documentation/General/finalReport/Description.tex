\chapter{Description of the developed functionalities}

In this section we will discuss the main functionalities of our system.
MoodCat can be split up into three main features.
These functionalities are the ability to play music, chatting in the rooms and classifying a song.
They will be described in the sections below.
But first we will give a small overview of the structure of the system.

\section{Introduction to MoodCat}
When a user enters the website he selects one or more moods (for example: happy, exciting, relaxed).
Then a list of rooms fitting that selection of moods will appear.
When clicking on a room a user joins, and he can listen and chat with other people in the room.
A user can vote if the song is wrongly classified for a room.
If this is the case, a small pop-up will appear and asks how the song should be classified instead.
MoodCat also has a gamification aspect where a user is served with songs not yet classified. The user then gets a snippet of the song (30 seconds at most) and is then asked for a rating.

\section{Listening to music}
MoodCat is developed as a music application and thus the ability to play music is very important.
Although, our music player is a bit different from a normal music player.
MoodCat works as a streaming service, and because of that the music is synchronized with the room the player is in.  
So you can only mute the sound, not alter its progression.
When the internet connection of the user goes down for a few seconds or the connection is slow, the music synchronises with the room as soon as possible.
In fact, a room in MoodCat can be compared to a radio station.

\section{Chatting}
In Moodcat you can chat with other people in the room.
In order to send a message the user needs to be logged in using SoundCloud.
At SoundCloud you can register with one click using your existing Facebook or Google account.

When a user is logged in, he can chat by typing a message in the "Type a message" bar.
The user sees the messages from other users, accompanied by their usernames.
The name of a user is the name a user has set in SoundCloud as their real name.

\section{Classifying}
When a user disagrees with the classification of a song, the user can give his own classification.
This is done with valence and arousal, which are terms to express the mood of a user.
To classify a user gets different buttons with an image of state.
These images are designed in such a way that it only can be seen as one state.
When a user plays the classifying game the user classifies the snippets with the same buttons as the normal classification.

\par
The classification effects how the song is classified as a mood. When a song is downvoted by a lot of people in the room, the song will not be playied in that room anymore.
It can be scheduld in another room though, according to its  ajusted classification.
This way the system improves over time.
When a user can earn points by classifying songs and is therefore encouraged to train the system.
The scores can be found on the leaderboard. 
For classifying an unrated song a user gets more points than a normal classification.   
This is done to stimulate the user to classify more songs, and thus grow MoodCat's collection.