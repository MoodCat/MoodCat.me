\chapter{Introduction}

The context project is a product in which bachelor students at the TU Delft get a task to develop a system in 10 weeks for a given problem.

The main problem our context had to face was to design the music services of tomorrow.
Our group had to design an application that would allow users to listen to music of a certain mood and chat with people in the same mood.\\

The requirements of the system is that users can select a mood and a room and then join a room according to their mood.
In the room they can get with each other, if they are logged in.
A user can classify a song according to mood of the song.
A user can also go the ranking game and classify unranked songs.
The user gets points for classifying a songs.\\


In this report we will explain MoodCat and how it is made.
First we will give an overview of the overall system, then we will reflect on the product and process.
Third we will explain the different functionalities of MoodCat and then we will explain then interaction between the user and our system.
Then we will evaluate the functionalities and tell about the features, lastly we will give an outlook in which we explain what we want to improve in the future.