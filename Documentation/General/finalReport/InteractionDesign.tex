\chapter{Human-Computer Interaction}

\section{Introduction}
This section describes the interaction between the user and the computer and how we measured this.
The user is very important for us, the user is the one that should use the application.
Our system is build for interaction between users and is improving by user interaction.
The interaction between users is the chatroom that our whole music application is build around.
The music itself is based on valence and arousal witch is classified by the user.
So, the more users our system has the better it will work.


%Beschrijving van de gebruiker en gebruiksituatie

%• Gebruikersverhalen: duidelijk maken wat mensen willen en doen (0)
%• Conceptuele scenario's: genereren van ideeën en vaststellen van requirements (0)
%• Concrete scenario’s: Gebruik maken van prototypes en voor evaluatie (1)
%• Use cases: voor documentatie en implementatie van interactie (3)

\section{The user and their situation}

\subsection{Use case scenarios}
Here are a few use case scenarios, we have made more.
Due to readability we decided to only write down the 3 most important use case scenarios.

\subsubsection{User wants to play music on MoodCat.me.}
\begin{itemize}
\item The user goes to moodcat.me in his browser
\item The user clicks on the mood he is in
\item The user clicks next
\item The user selects a room
\end{itemize}

\subsubsection{Whilst in a room the user wants to chat}
If the user is not logged in yet:
\begin{itemize}
\item Click on "login" in the top right corner
\item Click on the connect button
\end{itemize}
Everyone:
\begin{itemize}
\item The user clicks on the bar which says "type a message".
\item The user clicks "send"
\end{itemize}

\subsubsection{User wants to do the classify game}
\begin{itemize}
\item User clicks on "Rating game" in the header
\item User clicks a song
\item User listens to a song
\item User selects the appropriate buttons for valence and arousal
\item User clicks submit
\item User gets 6 points
\item User comes back to the menu to click a new song to classify
\end{itemize}


%Beschrijving van context inquiry
\section{Context inquiry}
% A contextual inquiry interview is usually structured as an approximately two-hour, one-on-one interaction in which the researcher watches the user do their normal activities and discusses what they see with the user.
We did a context inquiry with 5 other students a the tu delft.
All 5 of us watched a different person, but we were sitting in the same room.
They were separated by some free space, so they would not be influenced by each other.
We just asked them to play with the system and we would see how they do.
To see how they experienced the system is quite interesting, a user acts different in the first 5 minutes of system use.
Then the system is new and such, but later on they of course knew all the features.
These are two very important aspects.
First off all we want to bind the user to our system, so he will use it and then we want them to keep using the system.
So, that the user comes back often, because he likes the system.\\

%More about context inquiry

%Beschrijving van product design, claim analyze over gebruik ondersteuned met HCI literatuur
\section{Product design and claim analysis}
MoodCat is developed with the end-user as starting point, we tried to make the interface as intuitive as possible.
The interface should not contain a lot of information, so that the user remains the overview.
%What to say more

\subsection{Claim analysis}
MoodCat is developed, so that people can listen to music and chat with people who are in the same mood.
We have resourced this and gamers streaming on twitch is very popular, due to the twitch chat.
A record company Monstercat streams music on twitch, this is very popular and people seem to enjoy this.
We see this coming back in the interaction session with possible end-users, that they really enjoy the chat.


%Beschrijving van een bruikbaarheidsevaluatie uitgevoerd met het product
\section{Usability evaluation}
\subsection{Cognitive Walkthroughs}
During the design of the product we did multiple usability evaluations.
We have done quite a few Cognitive Walkthroughs, with changing possible end-users.
We done this with the thinking out loud protocol, so we always knew there they where stuck or did not fully understand something.
When you do not ask people to think out loud, they would sometimes hide some mistakes, which later on can be very important.
We gave the users the task to explore the functionalities of the system.
We got some feedback and found that very valuable.
Mainly, because in an early stage you know what goes wrong and right. 
So, anticipating on the results is therefore a lot easier than later on.
\subsection{GOMS}
At the end of our project we did a GOMS analysis.
We asked them to do some task and watched how quickly they preformed these tasks.
These where the tasks of the Cognitive Walktrough, only split into multiple sections.
We wanted to see where sometime went wrong. 
So, not that the user had some difficulties with finding a chatroom, but that they spent longer on the select a mood page than we would expect.
These more specific failures gave us more information to work with.

