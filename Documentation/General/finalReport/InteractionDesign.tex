\chapter{Human-Computer Interaction}

\section{Introduction}
In order to evaluate the usability of the system, we have organised various testing sessions with potential users.
In these testing sessions we used a think-aloud approach in order to gather user feedback.
First we will elaborate on the claim analysis and what knowledge we expect the user to have.
Secondly we will explain the setup of the testing sessions and what the outcomes were.

\section{Product design and claim analysis}
We have various claims on what we expect of the user.
In this section we will elaborate the 2 most important claims: user navigation and social interaction.

\subsection{User navigation}

The interface was designed to be minimalistic and intuitive.
The user should be able to navigate throughout the system without external guidelines or instructions.

We do assume the user is able to understand the ways on how to navigate from the mood selection to a room.
We have had several iterations of the state transitions to smooth out the user experience.
Initially, after selecting the moods, the list of rooms was added to the screen.
After evaluating these transitions with some users, we realized the transition was not clear enough.
Therefore we changed the behaviour to hide the moods and only show the list of rooms after a mood was selected.
This resulted in the users understanding the different steps in process of entering a room from a mood.

After changing this, we received additional feedback about the instructions.
The instructions at the various steps were unclear and could be worded differently.
We agreed that the instructions could be more fine-grained and decided to place little textboxes on the page.
These textboxes guide the user, if they require more information.
Therefore it does not bother users whom already know how to use the system, but will help beginners for which the instructions are aimed.

\subsection{Social interaction}

A room is defined as a collection of users listening to the same song at the same time.
Additionally we created a chat message system to let users interact with each other.

We claim that users enjoy or prefer to have social interaction with other users.
For a part of the potential userbase, listening to music is a social activity.
Good examples of these activities where this is apparant are festivals and pubs.
In these places, music is played and groups of people enjoy the music together.
It is not necessary that the persons in the group know each other, as experiencing music is a universal activity.

Our chat system provides the above group the possibility to interact with each other, in order to express their feelings.
However, the chat system does not bother the user group that does not require social interaction.
This group usually consists of people who listen to music passively; when they are doing other activities while enjoying music.
If you would use MoodCat passively, the interface is hidden for the users.
Therefore the user does not view the page and does not see the chat system either.

We have asked our testers explicitly if they appreciate the placement chat box.
In general it did not bother them.
The purpose of the system was clear and the usage was intuitive.
So far we have not received a single negative response regarding it and therefore we assume the usage is easy and the results succesful.
Additionally we saw various users immediately post messages to the rooms they were in, by checking our logs.

Concluding we think the general intention is clear and it is a succesful addition to our product.

\section{Context inquiry}
We've had quite a few people test MoodCat.me over the weeks. Of those, we chose 6 people to test our system extensively, under our supervision.\\

The general consensus was that on first join, it wasn't entirely clear what had to be done.
To remedy this, we've placed small texts to aid new users, without annoying returning users.
We've also placed a graphic to stimulate the users to log in with Soundcloud. After a few minutes, the system felt natural to the users.\\

Two of the testers mentioned that it wasn't perfectly clear which moods were selected and which weren't, so we've clarified that.
Most users also expected some form of feedback after voting, besides the button being disabled.\\

Virtually all other issues were known issues and already were planned for the next sprint(s).\\

From the first week, our goal was a clean, dependable interface.