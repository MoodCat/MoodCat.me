\chapter{Human-Computer Interaction}

\section{Introduction}
In order to evaluate the usability of the system, we have organised various testing sessions with potential users.
In these testing sessions we used a think-aloud approach in order to gather user feedback.
First we will elaborate on the claim analysis and what knowledge we expect the user to have.
Secondly we will explain the setup of the testing sessions and what the outcomes were.

\section{Product design and claim analysis}
We have various claims on what we expect of the user.
In this section we will elaborate the two most important claims: user navigation and social interaction.

\subsection{User navigation}

The interface was designed to be minimalistic and intuitive.
The user should be able to navigate throughout the system without external guidelines or instructions.

We do assume the user is able to understand the ways how to navigate from the mood selection to a room.
We had several iterations of the state transitions to smooth out the user experience.
After selecting the moods, rooms appeared on the screen.
In the evalution some users did not fully understand this and we realized the transition was not clear enough.
Therefore we changed the behaviour to hide the moods and only show the list of rooms after a mood was selected.
This resulted in the users understanding the different steps in process of entering a room from a mood.

After changing this, we received additional feedback about the instructions.
The instructions at the various steps were unclear and could be worded differently.
We agreed that the instructions could be more fine-grained and decided to place little textboxes on the page.
These textboxes guide the user, if they require more information.
Therefore it does not bother users whom already know how to use the system, but will help beginners for which the instructions are aimed.

\subsection{Social interaction}

Users meet in rooms, where music is playing.
Additionally we created a chat message system that allows users to interact.

We claim that users enjoy or prefer to have social interaction with other users.
For a part of the potential userbase, listening to music is a social activity.
Good examples of these activities where this social element shines are festivals and pubs.
In these places music is played and groups of people enjoy the music together.
It is not necessary that the persons in the group know each other, as experiencing music is a universal activity.

Our chat system provides the above group the possibility to interact with each other, in order to express their feelings.
However, the chat system does not bother the user group that does not require social interaction.
This group usually consists of people who listen to music passively; when they are doing other activities while enjoying music.
If the user uses MoodCat passively, there is the possibility to hide the  browser from the desktop.
Therefore the user does not view the page and does not see the chat system either.

We have asked our testers explicitly if they appreciate the placement chatbox.
In general it did not bother them.
The purpose of the system was clear and the usage was intuitive.
So far we have not received a single negative response regarding it and therefore we assume the usage is easy and the results succesful.
Additionally we saw various users immediately post messages to the rooms they were in, by checking our logs.

Concluding we think the general intention is clear and it is a successful addition to our product.

\section{Context inquiry}
Moodcat has been used by a group of testers.
Additionally we supervised 6 testers who tested the system extensively.
In this section we will explain what the outcome was of the 6 think-out-loud testing sessions.

The first impressions were positive.
Everyone liked the user interface and the initial look-and-feel.
After everyone examined the homepage, the next goal was to go in a room.
Not all the testers were certain what to do at this point and required additional guiding.
Therefore we decided to add extra text on the various pages to explain how to reach the next step.
Secondly, the login system was not really noticeable at the beginning.
We had to explicitly point out that users have the ability to log in to our system.
In order to stimulate logging in and using the system by its full potential, we added a guiding arrow to point out the login button.

Besides the initial navigation problems, the testers did not have more trouble using and clicking through the system.
The purpose of the chatbox was clear, even though we did not provide explicit explanations.
Together with the song information, the general consensus was that the room was nicely  and intuitively designed.

Two of the testers mentioned that it was not clear which moods they selected and which they did not.
This was mostly related to the fact that the next button was placed as it were a mood.
We changed the behaviour of the next button to only show when moods are selected.
This animation implicitly guides users to click the next button when they are satisfied with their mood choices.

Most users also expected visual confirmation that the vote they casted had been received.
Initially this was only a disabled button.
We therefore added color changes based on the type of vote, to not only emphasize the succesful vote, but also to remind the user which vote they casted.

Apart from the above points, some small issues were discovered and have been fixed.
They were mostly related to sizing issues on the various platforms we tested on.

Concluding the developed system is well-received.
The only ambiguities that became apparent were related to navigation and guidance of the user.
Additions of small texts and changes in transitions have solved these issues.