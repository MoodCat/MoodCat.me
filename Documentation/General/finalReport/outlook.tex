\chapter{Outlook}
During the project, we developed several improvements to the system.
These improvements could not have been implemented due to time constraints, but the system designed in such a way that support for these features can be easily achieved.

%\begin{enumerate}
\label{outlook-neural}
\section{Song and room suggestions}
Our initial selling point was a neural-network based suggestion system.
A neural-network is a system that can classify items based on features and output in one (or few) dimensions.
In order to train a neural-network, you also need pre-determined labels/expected values for the dimensions.
In our case, the neural-network should output \gls{valence} and \gls{arousal}.
However, existing datasets did not contain those values.
How to obtain the valence and arousal of song-features is currently researched by various universities, but no method has been succesful yet.
Therefore we were unable to train and use our network.

At the moment users can classify songs and provide the system the expected valence and arousal.
The weights of the nodes in a network can be trained using the user-data gathered over time, which makes it possible to integrate a network in the future.

\section{Message polling versus broadcasted events}
%\item
Currently all room interactions are based on a status-poll structure between the frontend and backend.
The backend provides the song that is currently playing and the latest chat messages which the frontend uses to quickly check their status.
This unnecessarily stresses the network connection, as most calls do not contain additional information the frontend did not have yet.

To reduce the network usage, the polling-structure can be replaced with socket events.
Therefore only when there are updates, packets are sent between the frontend and backend.
This also makes changes more instant as the clients will immediately get notified, instead of on regular intervals.

\section{User interface}
%\item
There are various possible improvements to the user interface.

\begin{enumerate}
\item At the moment of writing, the room selection page contains a list of rooms.
This list of rooms is sorted by relevance, but there is no indication on how relevant a certain room is.

Our clients indicated a more suitable representation would be to display the select rooms in a 2D space.
This space would implicitly represent the inner vector structure used in the backend.
To indicate the selected moods and the corresponding vector, a point is marked in the space.
Using this represenation, relevant rooms are placed more closely to the point, whereas less relevant rooms are placed further away. The room selection page can be easily changed to suit this new representation.

\item The up-vote and down-vote buttons are at the moment placed in the top bar.
The initial philosophy was that while a song is playing, the user should be able to express his/her feelings.
Since the currently played song is in the top bar, the natural decision was to place the ote buttons next to it.

After gathering user feedback, one of the outcomes of the tests was that the purposes of the buttons was unknown.
Some prototypes can be developed with different placements of buttons which can be tested on a set of users.
Then the users can vote on the most intuitive solution.
\end{enumerate}
%\end{enumerate}
