\chapter{Neural Networks}

One of the bigger issues we had to deal with in MoodCat was to classify songs
to moods. The initial plan was to create a neural network that could do this.

Some of the MoodCat team members have followed a course about computational intelligence,
in which different ways to use learning algorithms to solve different kinds of problems.
One widely used technique to classify objects by using the objects' features as inputs
is the use of neural networks.

Neural networks consist of an input layer, an output layer and a number of hidden layers, 
and each layer consists of a number of nodes.

The nodes in the input layer are fed with 'features', if we take classifying songs as example
this would be all kinds of information of the songs that can be represented as some form
of numeric metrics. The amount of BPM of a song could be a metric, or the overall loudness
of songs. The key of a song is not a good metric because it cannot be represented as a scale,
e.g. music in A is not more like music in B than like music in G, while in numeric form A would be closer to B and on the other end of G.
However, the network could have an input for each key: if a song is written in A, the A input is 1 and the other key inputs
are 0.

The nodes in the hidden layers have nodes with weights for each node in the next layer,
and the output layers would have a node for each mood, the output nodes output a number representing how likely the song
is to correspond with that mood.

The weights in the network are initialized randomly, but trained by feeding training data to the network.
The training data consists of a list of songs along with their respective moods. When the network rightly classifies
the mood of a song, the weights are updated towards this outcome and vice versa. This way, the neural
network 'learns' how features relate with eachother, how they affect moods, and eventually how to classify songs to moods.

Sadly, we could not find a proper set of songs with their moods and features that could be linked to the features in our song database.
If we would have found such set we could have used it as training data for our network, have classified all our songs with the
trained network and so we could have avoided a cold start for the classification of the songs.
To compensate for this, we use the rating game as way to let our users classify the songs, and let them
earn points in the process.

\begin{figure}
	\centering
	\def\svgwidth{2in}
	\input{NeuralNetwork.pdf_tex}
\end{figure}
