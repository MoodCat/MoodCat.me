\chapter{Neural Networks}
\label{appendix-neural-networks}

One of the bigger issues we had to deal with in MoodCat was to classify the mood of a song.
The initial plan was to create a neural network that could do this.

Some of the MoodCat team members have followed a course Computational Intelligence,
where they learned different learning algorithms to solve different kinds of problems.
One widely used technique to classify objects by using the objects' features as inputs for the neural network.

\section{Structure of the neural network}
Neural networks consist of an input layer, an output layer and a number of hidden layers.
Each layer consists of a number of nodes (See also figure \ref{neural-network-graph}).
The nodes in the input layer are fed with \textit{features}.
For the classification of songs these would be all sorts of information of the song that can be represented as some form of numeric metrics, for example: the amount of beats per minute (BPM), the average loudness or tonality of the song.
The key of a song is not an obvious metric as it cannot be represented on a linear scale: music in A has not more in common with B than C or even G, while on a linear scale A would deviantly have more in common with B than G.
The network could however map the input for each key: if a song is written in A, the A input is 1 and the other key inputs are 0.

The nodes in the hidden layers have nodes with weights for each node in the next layer,
and the output layers would have a node for each mood, the output nodes output a number representing how likely the song
is to correspond with that mood.

The weights in the network are initialized randomly and trained by feeding training data to the network.
The training data consists of a list of songs along with their respective moods. When the classification is correct, the network doesn't do anything. However if the classification is incorrect, the weights are updated based on error propagation.
This way, the neural
network \textit{learns} how features relate, how they affect moods, and eventually how to classify songs to moods.

\section{Dataset for the neural network}
Sadly there was no dataset that contained either the song or features of songs and their corresponding moods.
Therefore we were unable to train the network with the provided data.
If we would have been able to train the network, we could have avoided the cold start problem the system currently suffers.
To compensate for this problem, we have designed the rating game.
The rating game let our users classify the songs.
In order to stimulate this behaviour, we let them award them with points.

\begin{figure}
    \caption{Neural network}
    \label{neural-network-graph}
	\centering
	\def\svgwidth{2in}
	\input{NeuralNetwork.pdf_tex}
\end{figure}
