\usepackage[toc,style=treenoname,order=word,subentrycounter]{glossaries}
\makeglossaries

\newglossaryentry{HTML}{
  name= HTML,
  description={HyperText Markup Language is the standard markup language used to create web pages}
}

\newglossaryentry{CSS}{
  name= CSS,
  description={Cascading Style Sheets is a style sheet language used for describing the look and formatting of a document written in a markup language}
}

\newglossaryentry{JS}{
  name= Javascript,
  description={A dynamic programming language. It is most commonly used as part of web browsers, whose implementations allow client-side scripts to interact with the user, control the browser, communicate asynchronously, and alter the document content that is displayed}
}

\newglossaryentry{HTTP}{
  name= HTTP,
  description={The Hypertext Transfer Protocol is an application protocol for distributed, collaborative, hypermedia information systems}
}

\newglossaryentry{JSON}{
  name= JSON,
  description={JavaScript Object Notation, is an open standard format that uses human-readable text to transmit data objects consisting of attribute–value pairs. It is used primarily to transmit data between a server and web application, as an alternative to XML}
}

\newglossaryentry{SQL}{
  name= SQL,
  description={Structured Query Language  is a special-purpose programming language designed for managing data held in a relational database management system (RDBMS), or for stream processing in a relational data stream management system (RDSMS)}
}

\newglossaryentry{valence}{
  name= valence,
  description={The amount of attraction you have towards a certain event}
}

\newglossaryentry{arousal}{
  name= arousal,
  description={How calm or excited you are}
}

\newglossaryentry{rtree}{
  name={R-tree},
  plural={R-trees},
  description={Data structure explained in appendix \ref{appendix:rtree}}
}

\newglossaryentry{kdtree}{
  name={KD-tree},
  plural={KD-trees},
  description={Data structure used for spatial querying, optimized by a recursive structure using buckets defined by bounding boxes}
}

\newglossaryentry{MVC}{
  name={MVC},
  description={Model–view–controller (MVC) is a software architectural pattern for implementing user interfaces.}
}

\newglossaryentry{MVVM}{
  name={MVVM},
  description={Model View ViewModel (MVVM) is an architectural pattern for software development.}
}

\newglossaryentry{angularjs}{
  name={AngularJS},
  description={AngularJS is an open-source web application framework maintained by Google and by a community of individual developers and corporations to address many of the challenges encountered in developing single-page applications. It aims to simplify both the development and the testing of such applications by providing a framework for client-side model–view–controller (MVC) and model-view-viewmodel (MVVM) architectures, along with components commonly used in rich Internet applications.}
}

\newglossaryentry{REST}{
  name={REST},
  description={Representational State Transfer (REST) is a software architecture style consisting of guidelines and best practices for creating scalable web services. REST was introduced by \citeauthor{rest}}
}

\newglossaryentry{data-binding}{
 name={data binding},
 description={Data binding is the process that establishes a connection between the application UI (User Interface) and business logic.}
 }
 
\newglossaryentry{ORM}{
 name={ORM},
 description={Object-relational mapping (ORM, O/RM, and O/R mapping) in computer science is a programming technique for converting data between incompatible type systems in object-oriented programming languages.}
 }
 
\newglossaryentry{Hibernate}{
 name={Hibernate},
 description={Hibernate ORM (Hibernate in short) is an object-relational mapping framework for the Java language, providing a framework for mapping an object-oriented domain model to a traditional relational database.}
 }


\newglossaryentry{Postgres}{
 name={PostgreSQL},
 description={PostgreSQL, often simply Postgres, is an object-relational database management system (ORDBMS) with an emphasis on extensibility and on standards-compliance.}
 }


\newglossaryentry{JAXRS}{
 name={JAX-RS},
 description={JAX-RS: Java API for RESTful Web Services (JAX-RS) is a Java programming language API that provides support in creating web services according to the Representational State Transfer (REST) architectural pattern.}
 }

\newglossaryentry{DI}{
 name={dependency injection},
 description={Dependency injection is a software design pattern that implements inversion of control. The responsibility for locating or constructing dependencies is expressly separated from code that would take responsibility for using those dependencies.}
 }
 
 \newglossaryentry{hibernate-spatial}{
 name={Hibernate Spatial},
 description={Hibernate Spatial is a generic extension to Hibernate for handling geographic data.}
 }
 
 \newglossaryentry{H2}{
 name={H2},
 description={H2 is a relational database management system written in Java. It can be embedded in Java applications or run in the client-server mode.}
 }

 \newglossaryentry{GeoDB}{
 name={GeoDB},
 description={GeoDB is a spatial extension of H2, the Java SQL database.}
 }
 
 \newglossaryentry{polling}{
 name={polling},
 description={Polling, or polled operation, in computer science, refers to actively sampling the status of an external device by a client program as a synchronous activity.}
 }

