\chapter{Reflection on the product and process from a software engineering perspective}
In this chapter, we discuss the software engineering perspectives of MoodCat.
First, we discuss our developing process.
Secondly, the general structure and architecture of the product is explained.
Then we will take a closer look at the used and implemented design paterns.
After that we will elaborate the choices to increase code readability and maintainability.
Finally, the most important libraries and languages we used will be listed.

\label{sec:product-architecture}
\section{Product Architecture}
The architecture of MoodCat is very similar compared to existing websites.
We're using a client-server architecture.
The frontend (client) is a AngularJS web application, whereas the backend (server) is a Java API based on JAX-RS\footnote{The implementation we use is RestEASY, which can be found on \url{http://resteasy.jboss.org/}}. 
The backend communicates with a persistence server in the form of a PostgreSQL database.

\section{Design Patterns}
Due to the usage of various libraries and frameworks, we use quite some design patterns.
Additionally, we did design our backend system very methodically, using the model-view-controller (MVC) pattern.
The API classes define the view of the application (in the form of JSON responses) which is fetched by the frontend to be used by the client.
The "backend" package contains the controllers of the system.
The entities in the "database" package define our models.

In the frontend we use a slightly different structure, most commonly referred to MVVM (Model, View, ViewModel).
The API calls are the models obtaining the data.
The controllers and services are the viewmodels, which not only show but also alter the content.
And lastly the HTML templates are the views showing the website to the user.

A few of the design patterns we used are:
\begin{commalist}
\item Facades
\item Proxies
\item Singletons
\item Factories
\item Dependency Injection
\item MVC
\item MVVM
\item resource pooling
\end{commalist}.

\section{Code readability / maintainability}
In our efforts to create a codebase that not only fit our needs, but is readable and maintainable by others as well, we valued clean, short code greatly.
Methods are kept short and simple, fill only one purpose and have descriptive names. 

We used the new technologies in Java 8 to it's full potential, substituting for loops and (nested) if statements for lambdas and anonymous classes wherever applicable.

In order to ease the communication with the database, we build the SQL commands using QueryDSL, which implements the builder pattern.
Added benefit was the type checking capability of this pattern, decreasing the risk of writing incorrect queries and making sure when making updates to entities, queries remain correct.

Google Guice allowed us to build our application like LEGO\texttrademark : composition at its best.
Guice is a Dependency Injection framework which allows looser coupling between classes and packages.
Responsibilities are split better and rapid prototyping is possible without having to worry how to pass dependencies around in the project.

Lastly, the MVC-structure enabled us to change inner workings of the backend without breaking the API contract with the frontend and/or without breaking the database schema.

\section{SCRUM}
We worked with the SCRUM-methodology for the second time.
Agile methodologies were known to most of us before this project, but no one had really used it consistently, aside from the basics of SCRUM during the SEM course.
We did some research and found a really nice guide from Atlassian\footnote{https://www.atlassian.com/agile}.
We learned about the different roles of SCRUM: the SCRUM Master, the Product Owner and the development team, as well as how to manage meetings in a proper manner.
We decided to have daily SCRUM meetings at 09.00 o'clock every morning.
During this meeting, we would use our Waffle\footnote{http://waffle.io} board to keep track of the groups progress.
Each member of the group had to answer 4 questions:
\begin{enumerate}
\item What have you done yesterday?
\item What are you going to do today?
\item Are you on schedule?
\item Do you need help on anything?
\end{enumerate}
This went surprisingly well!
In our first sprint we had a little trouble guessing how long an issue would take.
We discovered quickly in the following sprints that story points in Waffle helped us managing tickets and sprints.
We decided that 1 story point equals 2 hours of work.

SCRUM meetings were useful to keep an overview of the system, as well as help others out when they were stuck.
We did continue past 15 minutes a few times, but usually with a smaller group for a specific issue.

Overall, we're all convinced SCRUM is a great way to develop software, however we think that a sprint could better last 2 or 3 weeks instead of one.
The sprint planning and sprint reflection took up a considerable amount of time in the sprint and the weekly demo forced us to put a difficult issue aside more than once.

\section{Use of libaries and languages}
We have used quite a lot of languages and libraries.
To get a clear view of what belongs where, we split the system in our three components.

\subsection{Frontend}
The frontend for Moodcat is the website loaded into the user's browser.

\subsubsection{Languages}
To create a website, we use the standard web languages.
HTML for the DOM creation, javascript for client-side logic, DOM manipulation and communication with the backend API and CSS for styling.

\subsubsection{Libraries}
We decided to use AngularJS to increase the productivity of developing features on the frontend.
This is a framework built upon javascript and it deals exceptionally well with DOM updates (with 2-way data binding) and HTTP calls.
It also enables re-usable code with directives, a mechanism to create an on the fly webcomponent.
We actually extend AngularJS further with packages to suit our specific needs.

In order to generate CSS, we use preprocessor called SASS that eases the re-use of CSS rules and markup.

\subsection{Backend}

\subsubsection{Languages}
The backend is written in Java.
We've chosen Java because everyone in the team was most experienced in this language.

\subsubsection{Libraries}
To turn the application into a servlet and process HTTP requests, we use Jetty.
The RESTful API is implemented using RestEASY, a library that implements the interfaces defined by JAX-RS.
JAX-RS has a dependency injection module to ease the process of defining end-points.
However, we also wanted a more complete dependency injection framework.
Therefore we decided to use Guice for constructor injection, together with Guice-JPA for transaction support and Guice- assisted-inject for factories.

We're using Hibernate for ORM (Object-relational mapping).
This allows us to transparently "store" Java objects in the database.
Tables are generated according to the fields defined in the entities.

In order to quickly select songs or rooms from the database, based on their vector, we use a RTREE as spatial index.
In order to enable RTREEs in our database and backend, we use Hibernate Spatial.

In order to ease the development process, we use two more dependencies:
To simulate a postgres database during testing, we're using H2 (with Hatbox for spatial query support).
Besides that, we're using Project Lombok for automatic generation of getters, setters, hashcode and equals methods.
