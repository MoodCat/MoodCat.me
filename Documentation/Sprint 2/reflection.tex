\documentclass[11pt,a4paper,landscape]{article}
\usepackage{titling,graphicx,palatino,amsmath,enumerate,booktabs,listings,color}
\usepackage[none]{hyphenat}
\usepackage[parfill]{parskip}
\usepackage{pgfplots}
\usepackage{longtable}
\usepackage[top=2 cm, bottom= 2 cm, left=1 cm, right=1 cm]{geometry}
\usepackage{enumitem}
\begin{document}


\section*{Sprint reflection Plan 2}
Context : Multimedia services\\
Group : Multimedia services 1\\



\begin{table}[h]
\begin{tabular}{|p{3cm}|p{5.0cm}|p{3.0cm}|p{2.2cm}|p{2.2cm}|p{2.0cm}|p{5cm}}
\textbf{User Story} & \textbf{Task} & \textbf{Task assigned to} & \textbf{Estimated Effort per Task (Hours)} & \textbf{Actual Effort per Task (Hours)} & \textbf{Done} & \textbf{Notes}\\
As a developer I want to have a database to store data in.

&
\begin{enumerate}
\item Make database entities for Valence-arousal vector. 
\item Install a PostgreSQL instance on a vps.
\end{enumerate}

& 
\begin{enumerate}
\item Jan-Willem Gmelig Meyling
\item Jaap Heijligers
\end{enumerate}

& 
\begin{enumerate}
\item 2 hours
\item 2 hours
\end{enumerate}

&
\begin{enumerate}
\item 1 hour
\item 2 hours
\end{enumerate}

&
\begin{enumerate}
\item Yes
\item Yes
\end{enumerate}

&
\begin{enumerate}
\item 
\item Configuring the firewall caused some delay.
\end{enumerate}
\\

As a developer I want to make an algorithm that takes low-level music features and produces an valence-arousal vector.
&
\begin{enumerate}
\item Make a predictor implementation using the recommendation-library for the valence/arousal vector.
\item Make a test dataset with songs and our personal preferences to test our algorithm.
\item Test the predictor comparing our preference with the predicted preference.
\item Tweak the algorithm in order to produce reasonable vector-predictions.
\end{enumerate}

&
\begin{enumerate}
\item Tim van der Lippe
\item Eva Anker
\item Tim van der Lippe
\item Eva Anker
\end{enumerate}

&

\begin{enumerate}
\item 2 hours
\item 6 hours
\item 2 hours
\item 10 hours
\end{enumerate}

&
\begin{enumerate}
\item 4 hours
\item 4 hours
\item 2 hours
\item 3 hours
\end{enumerate}

&
\begin{enumerate}
\item Yes
\item Yes
\item Yes
\item No
\end{enumerate}
&
\begin{enumerate}
\item Took some extra time due to fitting the library to our needs.
\item 
\item Jan-Willem took over this issue.
\item Not finished due to an unforseen amount of documentation work. Has been transferred to next sprint.
\end{enumerate}

\end{tabular}
\end{table}
\begin{table}[h]
\begin{tabular}{|p{3cm}|p{5.0cm}|p{3.0cm}|p{2.2cm}|p{2.2cm}|p{2.0cm}|p{5cm}}
As a user I want to go to a room and chat with the people that are there with me.
I want to be able to save the url from a room and come back later.
&
\begin{enumerate}
\item Make the API on the backend for the chats.
\item Make the API on the backend for the rooms.
\item Change the frontend to communicate with the backend rather than to used mocked data for rooms and chats.
\end{enumerate}

&

\begin{enumerate}
\item Jaap Heijligers
\item Tim van der Lippe
\item Gijs Weterings
\end{enumerate}

&
\begin{enumerate}
\item 6 hours
\item 6 hours
\item 8 hours
\end{enumerate}

&
\begin{enumerate}
\item 3 hours
\item 6 hours
\item 9 hours
\end{enumerate}

&
\begin{enumerate}
\item Yes
\item Yes
\item Yes
\end{enumerate}
&
\begin{enumerate}
\item Swapped this issue and the next. Tim made this one. We found the communication between the frontend and backend tasks a bit troublesome, as we did not agree on a specific standard until later in the sprint.
\item Swapped this issue and the former. Jaap made this one. We found the communication between the frontend and backend tasks a bit troublesome, as we did not agree on a specific standard until later in the sprint.
\item Jan-Willem took over this issue.
\item Took slightly longer due to not clearly specifying the requirements first. We agreed to improve this in the future by sitting down together and agreeing on the interface first.
\end{enumerate}

\end{tabular}
\end{table}
\begin{table}[h]
\begin{tabular}{|p{3cm}|p{5.0cm}|p{3.0cm}|p{2.2cm}|p{2.2cm}|p{2.0cm}|p{5cm}}

As a teaching assistant I want to read the documention of the system.
&
\begin{enumerate}
\item Update product vision document with TA and information skills teacher feedback.
\item Write the architecture design document
\item Update the requirements document with the system changes of the first 2 sprints.
\end{enumerate} 
           
&
\begin{enumerate}
\item Tim van der Lippe
\item Jan-Willem Gmelig Meyling
\item Gijs Weterings
\end{enumerate}
           
&

\begin{enumerate}
\item 2 hours
\item 12 hours
\item 8 hours
\end{enumerate}

&
\begin{enumerate}
\item 5 hours
\item 15 hours
\item 13 hours
\end{enumerate}

&
\begin{enumerate}
\item Yes
\item Yes
\item Yes
\end{enumerate}
&
\begin{enumerate}
\item Jaap made the preface, we spend some more time to find some literature.
\item Took a bit longer due to extra feedback rounds.
\item This has been merged with the product planning document. The actual time is the time we took to create and update this full document, not just the MoSCoW bullets.
\end{enumerate} 
\end{tabular}
\end{table}

Problems we encountered:\\
The process of writing a document was not integrated with writing or updating the glossaries and references subsections. We need to ensure this will become a standard section of writing a document.\\
Reviewing Pull-Requests was vague because we did not properly assign storypoints for it and the estimated time for an issue was already exhausted before we went into reviewing and not even processing the review comments.
We have decided to create separate issues for reviewing the issues.\\
Issues were too big. For example, the Product Planning was mostly assigned to a single person. Therefore the others were just waiting for the document to be created. Ideally this was divided into separate subtasks and subissues that were assigned to different issues. In the next sprint we will create more and smaller issues, and spent more time in the planning to make these divisions.\\
We did not really do pair programming the way we wanted to. This is partly due to the large amount of documentation we had to do this week. \\
We agreed on keeping time better by editing the storypoints on waffle.io whenever an issue is done.




\end{document}